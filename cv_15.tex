%-----------------------------------------------------------------------------------------

\documentclass[a4paper,12pt]{memoir} % Fonte e tamanho do papel

%----------------------------------------------------------------------------------------

% pacotes usados
\usepackage[top=1cm,left=1cm,right=1cm,bottom=1cm]{geometry} % muda as margens

\usepackage[utf8]{inputenc} % para caracteres internacionais

\usepackage[T1]{fontenc} % Output para fontes internacionais

\usepackage{flowfram} % para o layout de duas colunas

\usepackage{enumitem} % para modificar listas

\usepackage{tikz} % para a regua vertical

% define a sidebar
\newflowframe{0.2\textwidth}{\textheight}{0pt}{0pt}[left]
\newlength{\LeftMainSep}
\setlength{\LeftMainSep}{0.2\textwidth}
\addtolength{\LeftMainSep}{1\columnsep}

\setlength{\columnsep}{\baselineskip} % define o espaço entre as colunas

% frame estatico para a linha vertical
\newstaticframe{1.5pt}{\textheight}{\LeftMainSep}{0pt}

% conteudo da linha vertical statica
\begin{staticcontents}{1}
	\hfill
	\tikz{\draw[line width=1pt,yshift=0](0,0) -- (0,\textheight);}
\hfill\mbox{}
\end{staticcontents}

% define a parte direita ( o principal )
\addtolength{\LeftMainSep}{1.5pt}
\addtolength{\LeftMainSep}{1\columnsep}
\newflowframe{0.7\textwidth}{\textheight}{\LeftMainSep}{0pt}[main01]

\pagestyle{empty} % tira a numeração de pagina
\setlength{\parindent}{0pt} % retira a identação de paragrafo
 % Inclui o documento que altera o layout e algumas configurações

% lateral \userinformation
{
	\begin{flushright}

	\textbf{\Large{Seu Nome Aqui}} \\			% Seu nome
	\small																% Uma fonte menor
	01/01/0000 \\													% Data de nascimento
	\url{seu@email.aqui} \\							  % Seu e-mail
	\url{www.seusite.aqui} \\						  % Seu site se tiver
	(11) 11111-1111 \\ 										% Seu número de telefone
	\vspace{1em} 													% Espaço em branco
	\textbf{Endereço} \\
	Número, Rua \\ 												% Número, Rua
	Apartamento, Número										% Apartamento, Número
	Cidade, Estado \\ 										% Cidade, Estado
	País \\ 															% Comenta se não achar necessario
	\vfill 																% espaço em branco para jogar o conteudo pra baixo
	\end{flushright}
}


\begin{document}
%----------------------------------------------------------------------------------------

\userinformation 	% printa a coluna da esquerda

\framebreak 			% fim da coluna da esquerda

\hfill \\   			% gambiarra para alinhar textos iniciais

%----------------------------------------------------------------------------------------

% Resumo

\textbf{\Large {Seu Nome Aqui}}

\section*{Resumo}

\paragraph
{
	Lorem ipsum dolor sit amet, consectetur adipiscing elit. Nullam in faucibus risus, ac hendrerit arcu. Vivamus at tortor ex. In quis magna tellus.
}

%----------------------------------------------------------------------------------------

% Formações

\section*{Formações}

\begin{itemize}
	\item Período ( Curso - Instituição, Cidade/UF )
	\item Período ( Curso - Instituição, Cidade/UF )
\end{itemize}

%----------------------------------------------------------------------------------------

% Conhecimentos

\section*{Conhecimentos}

\subsection*{Idiomas}

\begin{itemize}
	\item Nível ( Item1, Item2, ...)
	\item Nível ( Item1, Item2, ...)
\end{itemize}

\subsection*{Linguagens de programação}

\begin{itemize}
	\item Nível ( Item1, Item2, ...)
	\item Nível ( Item1, Item2, ...)
\end{itemize}

\subsection*{Linguagens de marcação}

\begin{itemize}
	\item Nível ( Item1, Item2, ...)
	\item Nível ( Item1, Item2, ...)
\end{itemize}

\subsection*{Miscelâneas}

\begin{itemize}
	\item Nível ( Item1, Item2, ...)
	\item Nível ( Item1, Item2, ...)
\end{itemize}

%----------------------------------------------------------------------------------------

\end{document}
